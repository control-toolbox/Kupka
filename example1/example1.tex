\documentclass[11pt]{article}

    \usepackage[breakable]{tcolorbox}
    \usepackage{parskip} % Stop auto-indenting (to mimic markdown behaviour)
    
    \usepackage{iftex}
    \ifPDFTeX
    	\usepackage[T1]{fontenc}
    	\usepackage{mathpazo}
    \else
    	\usepackage{fontspec}
    \fi

    % Basic figure setup, for now with no caption control since it's done
    % automatically by Pandoc (which extracts ![](path) syntax from Markdown).
    \usepackage{graphicx}
    % Maintain compatibility with old templates. Remove in nbconvert 6.0
    \let\Oldincludegraphics\includegraphics
    % Ensure that by default, figures have no caption (until we provide a
    % proper Figure object with a Caption API and a way to capture that
    % in the conversion process - todo).
    \usepackage{caption}
    \DeclareCaptionFormat{nocaption}{}
    \captionsetup{format=nocaption,aboveskip=0pt,belowskip=0pt}

    \usepackage[Export]{adjustbox} % Used to constrain images to a maximum size
    \adjustboxset{max size={0.9\linewidth}{0.9\paperheight}}
    \usepackage{float}
    \floatplacement{figure}{H} % forces figures to be placed at the correct location
    \usepackage{xcolor} % Allow colors to be defined
    \usepackage{enumerate} % Needed for markdown enumerations to work
    \usepackage{geometry} % Used to adjust the document margins
    \usepackage{amsmath} % Equations
    \usepackage{amssymb} % Equations
    \usepackage{textcomp} % defines textquotesingle
    % Hack from http://tex.stackexchange.com/a/47451/13684:
    \AtBeginDocument{%
        \def\PYZsq{\textquotesingle}% Upright quotes in Pygmentized code
    }
    \usepackage{upquote} % Upright quotes for verbatim code
    \usepackage{eurosym} % defines \euro
    \usepackage[mathletters]{ucs} % Extended unicode (utf-8) support
    \usepackage{fancyvrb} % verbatim replacement that allows latex
    \usepackage{grffile} % extends the file name processing of package graphics 
                         % to support a larger range
    \makeatletter % fix for grffile with XeLaTeX
    \def\Gread@@xetex#1{%
      \IfFileExists{"\Gin@base".bb}%
      {\Gread@eps{\Gin@base.bb}}%
      {\Gread@@xetex@aux#1}%
    }
    \makeatother

    % The hyperref package gives us a pdf with properly built
    % internal navigation ('pdf bookmarks' for the table of contents,
    % internal cross-reference links, web links for URLs, etc.)
    \usepackage{hyperref}
    % The default LaTeX title has an obnoxious amount of whitespace. By default,
    % titling removes some of it. It also provides customization options.
    \usepackage{titling}
    \usepackage{longtable} % longtable support required by pandoc >1.10
    \usepackage{booktabs}  % table support for pandoc > 1.12.2
    \usepackage[inline]{enumitem} % IRkernel/repr support (it uses the enumerate* environment)
    \usepackage[normalem]{ulem} % ulem is needed to support strikethroughs (\sout)
                                % normalem makes italics be italics, not underlines
    \usepackage{mathrsfs}
    

    
    % Colors for the hyperref package
    \definecolor{urlcolor}{rgb}{0,.145,.698}
    \definecolor{linkcolor}{rgb}{.71,0.21,0.01}
    \definecolor{citecolor}{rgb}{.12,.54,.11}

    % ANSI colors
    \definecolor{ansi-black}{HTML}{3E424D}
    \definecolor{ansi-black-intense}{HTML}{282C36}
    \definecolor{ansi-red}{HTML}{E75C58}
    \definecolor{ansi-red-intense}{HTML}{B22B31}
    \definecolor{ansi-green}{HTML}{00A250}
    \definecolor{ansi-green-intense}{HTML}{007427}
    \definecolor{ansi-yellow}{HTML}{DDB62B}
    \definecolor{ansi-yellow-intense}{HTML}{B27D12}
    \definecolor{ansi-blue}{HTML}{208FFB}
    \definecolor{ansi-blue-intense}{HTML}{0065CA}
    \definecolor{ansi-magenta}{HTML}{D160C4}
    \definecolor{ansi-magenta-intense}{HTML}{A03196}
    \definecolor{ansi-cyan}{HTML}{60C6C8}
    \definecolor{ansi-cyan-intense}{HTML}{258F8F}
    \definecolor{ansi-white}{HTML}{C5C1B4}
    \definecolor{ansi-white-intense}{HTML}{A1A6B2}
    \definecolor{ansi-default-inverse-fg}{HTML}{FFFFFF}
    \definecolor{ansi-default-inverse-bg}{HTML}{000000}

    % commands and environments needed by pandoc snippets
    % extracted from the output of `pandoc -s`
    \providecommand{\tightlist}{%
      \setlength{\itemsep}{0pt}\setlength{\parskip}{0pt}}
    \DefineVerbatimEnvironment{Highlighting}{Verbatim}{commandchars=\\\{\}}
    % Add ',fontsize=\small' for more characters per line
    \newenvironment{Shaded}{}{}
    \newcommand{\KeywordTok}[1]{\textcolor[rgb]{0.00,0.44,0.13}{\textbf{{#1}}}}
    \newcommand{\DataTypeTok}[1]{\textcolor[rgb]{0.56,0.13,0.00}{{#1}}}
    \newcommand{\DecValTok}[1]{\textcolor[rgb]{0.25,0.63,0.44}{{#1}}}
    \newcommand{\BaseNTok}[1]{\textcolor[rgb]{0.25,0.63,0.44}{{#1}}}
    \newcommand{\FloatTok}[1]{\textcolor[rgb]{0.25,0.63,0.44}{{#1}}}
    \newcommand{\CharTok}[1]{\textcolor[rgb]{0.25,0.44,0.63}{{#1}}}
    \newcommand{\StringTok}[1]{\textcolor[rgb]{0.25,0.44,0.63}{{#1}}}
    \newcommand{\CommentTok}[1]{\textcolor[rgb]{0.38,0.63,0.69}{\textit{{#1}}}}
    \newcommand{\OtherTok}[1]{\textcolor[rgb]{0.00,0.44,0.13}{{#1}}}
    \newcommand{\AlertTok}[1]{\textcolor[rgb]{1.00,0.00,0.00}{\textbf{{#1}}}}
    \newcommand{\FunctionTok}[1]{\textcolor[rgb]{0.02,0.16,0.49}{{#1}}}
    \newcommand{\RegionMarkerTok}[1]{{#1}}
    \newcommand{\ErrorTok}[1]{\textcolor[rgb]{1.00,0.00,0.00}{\textbf{{#1}}}}
    \newcommand{\NormalTok}[1]{{#1}}
    
    % Additional commands for more recent versions of Pandoc
    \newcommand{\ConstantTok}[1]{\textcolor[rgb]{0.53,0.00,0.00}{{#1}}}
    \newcommand{\SpecialCharTok}[1]{\textcolor[rgb]{0.25,0.44,0.63}{{#1}}}
    \newcommand{\VerbatimStringTok}[1]{\textcolor[rgb]{0.25,0.44,0.63}{{#1}}}
    \newcommand{\SpecialStringTok}[1]{\textcolor[rgb]{0.73,0.40,0.53}{{#1}}}
    \newcommand{\ImportTok}[1]{{#1}}
    \newcommand{\DocumentationTok}[1]{\textcolor[rgb]{0.73,0.13,0.13}{\textit{{#1}}}}
    \newcommand{\AnnotationTok}[1]{\textcolor[rgb]{0.38,0.63,0.69}{\textbf{\textit{{#1}}}}}
    \newcommand{\CommentVarTok}[1]{\textcolor[rgb]{0.38,0.63,0.69}{\textbf{\textit{{#1}}}}}
    \newcommand{\VariableTok}[1]{\textcolor[rgb]{0.10,0.09,0.49}{{#1}}}
    \newcommand{\ControlFlowTok}[1]{\textcolor[rgb]{0.00,0.44,0.13}{\textbf{{#1}}}}
    \newcommand{\OperatorTok}[1]{\textcolor[rgb]{0.40,0.40,0.40}{{#1}}}
    \newcommand{\BuiltInTok}[1]{{#1}}
    \newcommand{\ExtensionTok}[1]{{#1}}
    \newcommand{\PreprocessorTok}[1]{\textcolor[rgb]{0.74,0.48,0.00}{{#1}}}
    \newcommand{\AttributeTok}[1]{\textcolor[rgb]{0.49,0.56,0.16}{{#1}}}
    \newcommand{\InformationTok}[1]{\textcolor[rgb]{0.38,0.63,0.69}{\textbf{\textit{{#1}}}}}
    \newcommand{\WarningTok}[1]{\textcolor[rgb]{0.38,0.63,0.69}{\textbf{\textit{{#1}}}}}
    
    
    % Define a nice break command that doesn't care if a line doesn't already
    % exist.
    \def\br{\hspace*{\fill} \\* }
    % Math Jax compatibility definitions
    \def\gt{>}
    \def\lt{<}
    \let\Oldtex\TeX
    \let\Oldlatex\LaTeX
    \renewcommand{\TeX}{\textrm{\Oldtex}}
    \renewcommand{\LaTeX}{\textrm{\Oldlatex}}
    % Document parameters
    % Document title
    \title{example1}
    
    
    
    
    
% Pygments definitions
\makeatletter
\def\PY@reset{\let\PY@it=\relax \let\PY@bf=\relax%
    \let\PY@ul=\relax \let\PY@tc=\relax%
    \let\PY@bc=\relax \let\PY@ff=\relax}
\def\PY@tok#1{\csname PY@tok@#1\endcsname}
\def\PY@toks#1+{\ifx\relax#1\empty\else%
    \PY@tok{#1}\expandafter\PY@toks\fi}
\def\PY@do#1{\PY@bc{\PY@tc{\PY@ul{%
    \PY@it{\PY@bf{\PY@ff{#1}}}}}}}
\def\PY#1#2{\PY@reset\PY@toks#1+\relax+\PY@do{#2}}

\expandafter\def\csname PY@tok@w\endcsname{\def\PY@tc##1{\textcolor[rgb]{0.73,0.73,0.73}{##1}}}
\expandafter\def\csname PY@tok@c\endcsname{\let\PY@it=\textit\def\PY@tc##1{\textcolor[rgb]{0.25,0.50,0.50}{##1}}}
\expandafter\def\csname PY@tok@cp\endcsname{\def\PY@tc##1{\textcolor[rgb]{0.74,0.48,0.00}{##1}}}
\expandafter\def\csname PY@tok@k\endcsname{\let\PY@bf=\textbf\def\PY@tc##1{\textcolor[rgb]{0.00,0.50,0.00}{##1}}}
\expandafter\def\csname PY@tok@kp\endcsname{\def\PY@tc##1{\textcolor[rgb]{0.00,0.50,0.00}{##1}}}
\expandafter\def\csname PY@tok@kt\endcsname{\def\PY@tc##1{\textcolor[rgb]{0.69,0.00,0.25}{##1}}}
\expandafter\def\csname PY@tok@o\endcsname{\def\PY@tc##1{\textcolor[rgb]{0.40,0.40,0.40}{##1}}}
\expandafter\def\csname PY@tok@ow\endcsname{\let\PY@bf=\textbf\def\PY@tc##1{\textcolor[rgb]{0.67,0.13,1.00}{##1}}}
\expandafter\def\csname PY@tok@nb\endcsname{\def\PY@tc##1{\textcolor[rgb]{0.00,0.50,0.00}{##1}}}
\expandafter\def\csname PY@tok@nf\endcsname{\def\PY@tc##1{\textcolor[rgb]{0.00,0.00,1.00}{##1}}}
\expandafter\def\csname PY@tok@nc\endcsname{\let\PY@bf=\textbf\def\PY@tc##1{\textcolor[rgb]{0.00,0.00,1.00}{##1}}}
\expandafter\def\csname PY@tok@nn\endcsname{\let\PY@bf=\textbf\def\PY@tc##1{\textcolor[rgb]{0.00,0.00,1.00}{##1}}}
\expandafter\def\csname PY@tok@ne\endcsname{\let\PY@bf=\textbf\def\PY@tc##1{\textcolor[rgb]{0.82,0.25,0.23}{##1}}}
\expandafter\def\csname PY@tok@nv\endcsname{\def\PY@tc##1{\textcolor[rgb]{0.10,0.09,0.49}{##1}}}
\expandafter\def\csname PY@tok@no\endcsname{\def\PY@tc##1{\textcolor[rgb]{0.53,0.00,0.00}{##1}}}
\expandafter\def\csname PY@tok@nl\endcsname{\def\PY@tc##1{\textcolor[rgb]{0.63,0.63,0.00}{##1}}}
\expandafter\def\csname PY@tok@ni\endcsname{\let\PY@bf=\textbf\def\PY@tc##1{\textcolor[rgb]{0.60,0.60,0.60}{##1}}}
\expandafter\def\csname PY@tok@na\endcsname{\def\PY@tc##1{\textcolor[rgb]{0.49,0.56,0.16}{##1}}}
\expandafter\def\csname PY@tok@nt\endcsname{\let\PY@bf=\textbf\def\PY@tc##1{\textcolor[rgb]{0.00,0.50,0.00}{##1}}}
\expandafter\def\csname PY@tok@nd\endcsname{\def\PY@tc##1{\textcolor[rgb]{0.67,0.13,1.00}{##1}}}
\expandafter\def\csname PY@tok@s\endcsname{\def\PY@tc##1{\textcolor[rgb]{0.73,0.13,0.13}{##1}}}
\expandafter\def\csname PY@tok@sd\endcsname{\let\PY@it=\textit\def\PY@tc##1{\textcolor[rgb]{0.73,0.13,0.13}{##1}}}
\expandafter\def\csname PY@tok@si\endcsname{\let\PY@bf=\textbf\def\PY@tc##1{\textcolor[rgb]{0.73,0.40,0.53}{##1}}}
\expandafter\def\csname PY@tok@se\endcsname{\let\PY@bf=\textbf\def\PY@tc##1{\textcolor[rgb]{0.73,0.40,0.13}{##1}}}
\expandafter\def\csname PY@tok@sr\endcsname{\def\PY@tc##1{\textcolor[rgb]{0.73,0.40,0.53}{##1}}}
\expandafter\def\csname PY@tok@ss\endcsname{\def\PY@tc##1{\textcolor[rgb]{0.10,0.09,0.49}{##1}}}
\expandafter\def\csname PY@tok@sx\endcsname{\def\PY@tc##1{\textcolor[rgb]{0.00,0.50,0.00}{##1}}}
\expandafter\def\csname PY@tok@m\endcsname{\def\PY@tc##1{\textcolor[rgb]{0.40,0.40,0.40}{##1}}}
\expandafter\def\csname PY@tok@gh\endcsname{\let\PY@bf=\textbf\def\PY@tc##1{\textcolor[rgb]{0.00,0.00,0.50}{##1}}}
\expandafter\def\csname PY@tok@gu\endcsname{\let\PY@bf=\textbf\def\PY@tc##1{\textcolor[rgb]{0.50,0.00,0.50}{##1}}}
\expandafter\def\csname PY@tok@gd\endcsname{\def\PY@tc##1{\textcolor[rgb]{0.63,0.00,0.00}{##1}}}
\expandafter\def\csname PY@tok@gi\endcsname{\def\PY@tc##1{\textcolor[rgb]{0.00,0.63,0.00}{##1}}}
\expandafter\def\csname PY@tok@gr\endcsname{\def\PY@tc##1{\textcolor[rgb]{1.00,0.00,0.00}{##1}}}
\expandafter\def\csname PY@tok@ge\endcsname{\let\PY@it=\textit}
\expandafter\def\csname PY@tok@gs\endcsname{\let\PY@bf=\textbf}
\expandafter\def\csname PY@tok@gp\endcsname{\let\PY@bf=\textbf\def\PY@tc##1{\textcolor[rgb]{0.00,0.00,0.50}{##1}}}
\expandafter\def\csname PY@tok@go\endcsname{\def\PY@tc##1{\textcolor[rgb]{0.53,0.53,0.53}{##1}}}
\expandafter\def\csname PY@tok@gt\endcsname{\def\PY@tc##1{\textcolor[rgb]{0.00,0.27,0.87}{##1}}}
\expandafter\def\csname PY@tok@err\endcsname{\def\PY@bc##1{\setlength{\fboxsep}{0pt}\fcolorbox[rgb]{1.00,0.00,0.00}{1,1,1}{\strut ##1}}}
\expandafter\def\csname PY@tok@kc\endcsname{\let\PY@bf=\textbf\def\PY@tc##1{\textcolor[rgb]{0.00,0.50,0.00}{##1}}}
\expandafter\def\csname PY@tok@kd\endcsname{\let\PY@bf=\textbf\def\PY@tc##1{\textcolor[rgb]{0.00,0.50,0.00}{##1}}}
\expandafter\def\csname PY@tok@kn\endcsname{\let\PY@bf=\textbf\def\PY@tc##1{\textcolor[rgb]{0.00,0.50,0.00}{##1}}}
\expandafter\def\csname PY@tok@kr\endcsname{\let\PY@bf=\textbf\def\PY@tc##1{\textcolor[rgb]{0.00,0.50,0.00}{##1}}}
\expandafter\def\csname PY@tok@bp\endcsname{\def\PY@tc##1{\textcolor[rgb]{0.00,0.50,0.00}{##1}}}
\expandafter\def\csname PY@tok@fm\endcsname{\def\PY@tc##1{\textcolor[rgb]{0.00,0.00,1.00}{##1}}}
\expandafter\def\csname PY@tok@vc\endcsname{\def\PY@tc##1{\textcolor[rgb]{0.10,0.09,0.49}{##1}}}
\expandafter\def\csname PY@tok@vg\endcsname{\def\PY@tc##1{\textcolor[rgb]{0.10,0.09,0.49}{##1}}}
\expandafter\def\csname PY@tok@vi\endcsname{\def\PY@tc##1{\textcolor[rgb]{0.10,0.09,0.49}{##1}}}
\expandafter\def\csname PY@tok@vm\endcsname{\def\PY@tc##1{\textcolor[rgb]{0.10,0.09,0.49}{##1}}}
\expandafter\def\csname PY@tok@sa\endcsname{\def\PY@tc##1{\textcolor[rgb]{0.73,0.13,0.13}{##1}}}
\expandafter\def\csname PY@tok@sb\endcsname{\def\PY@tc##1{\textcolor[rgb]{0.73,0.13,0.13}{##1}}}
\expandafter\def\csname PY@tok@sc\endcsname{\def\PY@tc##1{\textcolor[rgb]{0.73,0.13,0.13}{##1}}}
\expandafter\def\csname PY@tok@dl\endcsname{\def\PY@tc##1{\textcolor[rgb]{0.73,0.13,0.13}{##1}}}
\expandafter\def\csname PY@tok@s2\endcsname{\def\PY@tc##1{\textcolor[rgb]{0.73,0.13,0.13}{##1}}}
\expandafter\def\csname PY@tok@sh\endcsname{\def\PY@tc##1{\textcolor[rgb]{0.73,0.13,0.13}{##1}}}
\expandafter\def\csname PY@tok@s1\endcsname{\def\PY@tc##1{\textcolor[rgb]{0.73,0.13,0.13}{##1}}}
\expandafter\def\csname PY@tok@mb\endcsname{\def\PY@tc##1{\textcolor[rgb]{0.40,0.40,0.40}{##1}}}
\expandafter\def\csname PY@tok@mf\endcsname{\def\PY@tc##1{\textcolor[rgb]{0.40,0.40,0.40}{##1}}}
\expandafter\def\csname PY@tok@mh\endcsname{\def\PY@tc##1{\textcolor[rgb]{0.40,0.40,0.40}{##1}}}
\expandafter\def\csname PY@tok@mi\endcsname{\def\PY@tc##1{\textcolor[rgb]{0.40,0.40,0.40}{##1}}}
\expandafter\def\csname PY@tok@il\endcsname{\def\PY@tc##1{\textcolor[rgb]{0.40,0.40,0.40}{##1}}}
\expandafter\def\csname PY@tok@mo\endcsname{\def\PY@tc##1{\textcolor[rgb]{0.40,0.40,0.40}{##1}}}
\expandafter\def\csname PY@tok@ch\endcsname{\let\PY@it=\textit\def\PY@tc##1{\textcolor[rgb]{0.25,0.50,0.50}{##1}}}
\expandafter\def\csname PY@tok@cm\endcsname{\let\PY@it=\textit\def\PY@tc##1{\textcolor[rgb]{0.25,0.50,0.50}{##1}}}
\expandafter\def\csname PY@tok@cpf\endcsname{\let\PY@it=\textit\def\PY@tc##1{\textcolor[rgb]{0.25,0.50,0.50}{##1}}}
\expandafter\def\csname PY@tok@c1\endcsname{\let\PY@it=\textit\def\PY@tc##1{\textcolor[rgb]{0.25,0.50,0.50}{##1}}}
\expandafter\def\csname PY@tok@cs\endcsname{\let\PY@it=\textit\def\PY@tc##1{\textcolor[rgb]{0.25,0.50,0.50}{##1}}}

\def\PYZbs{\char`\\}
\def\PYZus{\char`\_}
\def\PYZob{\char`\{}
\def\PYZcb{\char`\}}
\def\PYZca{\char`\^}
\def\PYZam{\char`\&}
\def\PYZlt{\char`\<}
\def\PYZgt{\char`\>}
\def\PYZsh{\char`\#}
\def\PYZpc{\char`\%}
\def\PYZdl{\char`\$}
\def\PYZhy{\char`\-}
\def\PYZsq{\char`\'}
\def\PYZdq{\char`\"}
\def\PYZti{\char`\~}
% for compatibility with earlier versions
\def\PYZat{@}
\def\PYZlb{[}
\def\PYZrb{]}
\makeatother


    % For linebreaks inside Verbatim environment from package fancyvrb. 
    \makeatletter
        \newbox\Wrappedcontinuationbox 
        \newbox\Wrappedvisiblespacebox 
        \newcommand*\Wrappedvisiblespace {\textcolor{red}{\textvisiblespace}} 
        \newcommand*\Wrappedcontinuationsymbol {\textcolor{red}{\llap{\tiny$\m@th\hookrightarrow$}}} 
        \newcommand*\Wrappedcontinuationindent {3ex } 
        \newcommand*\Wrappedafterbreak {\kern\Wrappedcontinuationindent\copy\Wrappedcontinuationbox} 
        % Take advantage of the already applied Pygments mark-up to insert 
        % potential linebreaks for TeX processing. 
        %        {, <, #, %, $, ' and ": go to next line. 
        %        _, }, ^, &, >, - and ~: stay at end of broken line. 
        % Use of \textquotesingle for straight quote. 
        \newcommand*\Wrappedbreaksatspecials {% 
            \def\PYGZus{\discretionary{\char`\_}{\Wrappedafterbreak}{\char`\_}}% 
            \def\PYGZob{\discretionary{}{\Wrappedafterbreak\char`\{}{\char`\{}}% 
            \def\PYGZcb{\discretionary{\char`\}}{\Wrappedafterbreak}{\char`\}}}% 
            \def\PYGZca{\discretionary{\char`\^}{\Wrappedafterbreak}{\char`\^}}% 
            \def\PYGZam{\discretionary{\char`\&}{\Wrappedafterbreak}{\char`\&}}% 
            \def\PYGZlt{\discretionary{}{\Wrappedafterbreak\char`\<}{\char`\<}}% 
            \def\PYGZgt{\discretionary{\char`\>}{\Wrappedafterbreak}{\char`\>}}% 
            \def\PYGZsh{\discretionary{}{\Wrappedafterbreak\char`\#}{\char`\#}}% 
            \def\PYGZpc{\discretionary{}{\Wrappedafterbreak\char`\%}{\char`\%}}% 
            \def\PYGZdl{\discretionary{}{\Wrappedafterbreak\char`\$}{\char`\$}}% 
            \def\PYGZhy{\discretionary{\char`\-}{\Wrappedafterbreak}{\char`\-}}% 
            \def\PYGZsq{\discretionary{}{\Wrappedafterbreak\textquotesingle}{\textquotesingle}}% 
            \def\PYGZdq{\discretionary{}{\Wrappedafterbreak\char`\"}{\char`\"}}% 
            \def\PYGZti{\discretionary{\char`\~}{\Wrappedafterbreak}{\char`\~}}% 
        } 
        % Some characters . , ; ? ! / are not pygmentized. 
        % This macro makes them "active" and they will insert potential linebreaks 
        \newcommand*\Wrappedbreaksatpunct {% 
            \lccode`\~`\.\lowercase{\def~}{\discretionary{\hbox{\char`\.}}{\Wrappedafterbreak}{\hbox{\char`\.}}}% 
            \lccode`\~`\,\lowercase{\def~}{\discretionary{\hbox{\char`\,}}{\Wrappedafterbreak}{\hbox{\char`\,}}}% 
            \lccode`\~`\;\lowercase{\def~}{\discretionary{\hbox{\char`\;}}{\Wrappedafterbreak}{\hbox{\char`\;}}}% 
            \lccode`\~`\:\lowercase{\def~}{\discretionary{\hbox{\char`\:}}{\Wrappedafterbreak}{\hbox{\char`\:}}}% 
            \lccode`\~`\?\lowercase{\def~}{\discretionary{\hbox{\char`\?}}{\Wrappedafterbreak}{\hbox{\char`\?}}}% 
            \lccode`\~`\!\lowercase{\def~}{\discretionary{\hbox{\char`\!}}{\Wrappedafterbreak}{\hbox{\char`\!}}}% 
            \lccode`\~`\/\lowercase{\def~}{\discretionary{\hbox{\char`\/}}{\Wrappedafterbreak}{\hbox{\char`\/}}}% 
            \catcode`\.\active
            \catcode`\,\active 
            \catcode`\;\active
            \catcode`\:\active
            \catcode`\?\active
            \catcode`\!\active
            \catcode`\/\active 
            \lccode`\~`\~ 	
        }
    \makeatother

    \let\OriginalVerbatim=\Verbatim
    \makeatletter
    \renewcommand{\Verbatim}[1][1]{%
        %\parskip\z@skip
        \sbox\Wrappedcontinuationbox {\Wrappedcontinuationsymbol}%
        \sbox\Wrappedvisiblespacebox {\FV@SetupFont\Wrappedvisiblespace}%
        \def\FancyVerbFormatLine ##1{\hsize\linewidth
            \vtop{\raggedright\hyphenpenalty\z@\exhyphenpenalty\z@
                \doublehyphendemerits\z@\finalhyphendemerits\z@
                \strut ##1\strut}%
        }%
        % If the linebreak is at a space, the latter will be displayed as visible
        % space at end of first line, and a continuation symbol starts next line.
        % Stretch/shrink are however usually zero for typewriter font.
        \def\FV@Space {%
            \nobreak\hskip\z@ plus\fontdimen3\font minus\fontdimen4\font
            \discretionary{\copy\Wrappedvisiblespacebox}{\Wrappedafterbreak}
            {\kern\fontdimen2\font}%
        }%
        
        % Allow breaks at special characters using \PYG... macros.
        \Wrappedbreaksatspecials
        % Breaks at punctuation characters . , ; ? ! and / need catcode=\active 	
        \OriginalVerbatim[#1,codes*=\Wrappedbreaksatpunct]%
    }
    \makeatother

    % Exact colors from NB
    \definecolor{incolor}{HTML}{303F9F}
    \definecolor{outcolor}{HTML}{D84315}
    \definecolor{cellborder}{HTML}{CFCFCF}
    \definecolor{cellbackground}{HTML}{F7F7F7}
    
    % prompt
    \makeatletter
    \newcommand{\boxspacing}{\kern\kvtcb@left@rule\kern\kvtcb@boxsep}
    \makeatother
    \newcommand{\prompt}[4]{
        \ttfamily\llap{{\color{#2}[#3]:\hspace{3pt}#4}}\vspace{-\baselineskip}
    }
    

    
    % Prevent overflowing lines due to hard-to-break entities
    \sloppy 
    % Setup hyperref package
    \hypersetup{
      breaklinks=true,  % so long urls are correctly broken across lines
      colorlinks=true,
      urlcolor=urlcolor,
      linkcolor=linkcolor,
      citecolor=citecolor,
      }
    % Slightly bigger margins than the latex defaults
    
    \geometry{verbose,tmargin=1in,bmargin=1in,lmargin=1in,rmargin=1in}
    
    

\begin{document}
    
    \maketitle
    
    

    
    \hypertarget{example-1}{%
\section{Example 1}\label{example-1}}

\hypertarget{problem}{%
\subsection{Problem}\label{problem}}

\[(OCP)_1\left\{\begin{array}{l}
Min \int_0^{t_f}x\sqrt{1+u^2}dt\\
\dot{x} = u\\
u\in\R
\end{array}
\right.\]

    \begin{tcolorbox}[breakable, size=fbox, boxrule=1pt, pad at break*=1mm,colback=cellbackground, colframe=cellborder]
\prompt{In}{incolor}{11}{\boxspacing}
\begin{Verbatim}[commandchars=\\\{\}]
\PY{c}{\PYZsh{} Packages}

\PY{k}{using} \PY{n}{Pkg}
\PY{n}{Pkg}\PY{o}{.}\PY{n}{activate}\PY{p}{(}\PY{l+s}{\PYZdq{}}\PY{l+s}{.}\PY{l+s}{\PYZdq{}}\PY{p}{)}
\PY{c}{\PYZsh{}}
\PY{k}{using} \PY{n}{OptimalControl}
\PY{k}{using} \PY{n}{LinearAlgebra}
\PY{k}{using} \PY{n}{ForwardDiff}
\PY{k}{using} \PY{n}{DifferentialEquations}
\PY{k}{using} \PY{n}{Roots}     \PY{c}{\PYZsh{} solve an equation f(x)=0 where f is from R to R}
\PY{c}{\PYZsh{} using MINPACK \PYZsh{} NLE solver }
\PY{c}{\PYZsh{} using NLsolve}
\PY{k}{using} \PY{n}{LaTeXStrings}

\PY{k}{using} \PY{n}{Test}
\end{Verbatim}
\end{tcolorbox}

    \begin{Verbatim}[commandchars=\\\{\}]
\textcolor{ansi-green-intense}{\textbf{  Activating}} project at `\textasciitilde{}/control-toolbox/indirect`
    \end{Verbatim}

    \hypertarget{control-toolbox-definition-of-the-problem}{%
\subsection{Control-toolbox definition of the
problem}\label{control-toolbox-definition-of-the-problem}}

    \begin{tcolorbox}[breakable, size=fbox, boxrule=1pt, pad at break*=1mm,colback=cellbackground, colframe=cellborder]
\prompt{In}{incolor}{12}{\boxspacing}
\begin{Verbatim}[commandchars=\\\{\}]
\PY{n}{t0} \PY{o}{=} \PY{l+m+mi}{0}
\PY{n}{tf} \PY{o}{=} \PY{l+m+mi}{2}
\PY{n}{x0} \PY{o}{=} \PY{l+m+mi}{1}
\PY{n}{xf} \PY{o}{=} \PY{l+m+mi}{1}
\PY{n+nd}{@def} \PY{n}{ocp} \PY{k}{begin}
    \PY{n}{t} \PY{o}{∈} \PY{p}{[} \PY{n}{t0}\PY{p}{,} \PY{n}{tf} \PY{p}{]}\PY{p}{,} \PY{n}{time}
    \PY{n}{x} \PY{o}{∈} \PY{n}{R}\PY{p}{,} \PY{n}{state}
    \PY{n}{u} \PY{o}{∈} \PY{n}{R}\PY{p}{,} \PY{n}{control}
    \PY{n}{x}\PY{p}{(}\PY{n}{t0}\PY{p}{)} \PY{o}{==} \PY{n}{x0}
    \PY{n}{x}\PY{p}{(}\PY{n}{tf}\PY{p}{)} \PY{o}{==} \PY{n}{xf}
    \PY{n}{ẋ}\PY{p}{(}\PY{n}{t}\PY{p}{)} \PY{o}{==}  \PY{n}{u}\PY{p}{(}\PY{n}{t}\PY{p}{)}
    \PY{n}{∫}\PY{p}{(}\PY{n}{x}\PY{p}{(}\PY{n}{t}\PY{p}{)}\PY{o}{*}\PY{p}{(}\PY{l+m+mi}{1} \PY{o}{+} \PY{n}{u}\PY{p}{(}\PY{n}{t}\PY{p}{)}\PY{o}{\PYZca{}}\PY{l+m+mi}{2}\PY{p}{)}\PY{o}{\PYZca{}}\PY{p}{(}\PY{l+m+mi}{1}\PY{o}{/}\PY{l+m+mi}{2}\PY{p}{)}\PY{p}{)} \PY{n}{→} \PY{n}{min}
\PY{k}{end}
\end{Verbatim}
\end{tcolorbox}

    
    \begin{verbatim}
t ∈ [t0, tf], time
x ∈ R, state
u ∈ R, control
x(t0) == x0
x(tf) == xf
ẋ(t) == u(t)
∫(x(t) * (1 + u(t) ^ 2) ^ (1 / 2)) → min

    \end{verbatim}

    
    \begin{Verbatim}[commandchars=\\\{\}]

┌───────┬───────┬─────────┬──────────┬───────────┬─────────────┐
│\textcolor{ansi-yellow}{ times }│\textcolor{ansi-yellow}{ state }│\textcolor{ansi-yellow}{ control }│\textcolor{ansi-yellow}{ dynamics }│\textcolor{ansi-yellow}{
objective }│\textcolor{ansi-yellow}{ constraints }│
├───────┼───────┼─────────┼──────────┼───────────┼─────────────┤
│    ✅ │    ✅ │      ✅ │       ✅ │        ✅ │          ✅ │
└───────┴───────┴─────────┴──────────┴───────────┴─────────────┘
    \end{Verbatim}

    \hypertarget{hamiltonian-flow}{%
\subsection{Hamiltonian flow}\label{hamiltonian-flow}}

The Hamiltonian is \[H(x,u,p) =  -x\sqrt{1+u^2} + up,\] and for
\(|p| < |x|\) the maximization of the Hamiltonian have the solution
\[u(x,p) = sign(x)p\sqrt{1. /(x^2-p^2)}.\] Then the true Hamiltonian is
\(H_r(x,p)=H(x,u(x,p),p)\).

We note \(z(t) = (x(t),p(t))\), then the hamiltonian flow is the
function \(\phi(.,z_0) = \phi(.,x_0,p_0)\) solution of the initial value
problem \[(IVP)\left\{
\begin{array}{l}
\dot{z} = \vec{H}(z) = \begin{pmatrix}
\dot{x}\\ \dot{p}
\end{pmatrix} =
\begin{pmatrix}
\dfrac{\partial H(z)}{\partial p}\\
-\dfrac{\partial H(z)}{\partial x}
\end{pmatrix}\\
z(0) = z_0 = (x_0,p_0).
\end{array}
    \right.\]

    \begin{tcolorbox}[breakable, size=fbox, boxrule=1pt, pad at break*=1mm,colback=cellbackground, colframe=cellborder]
\prompt{In}{incolor}{13}{\boxspacing}
\begin{Verbatim}[commandchars=\\\{\}]
\PY{c}{\PYZsh{} Compute the  Flow}

\PY{n}{u}\PY{p}{(}\PY{n}{x}\PY{p}{,} \PY{n}{p}\PY{p}{)} \PY{o}{=} \PY{n}{sign}\PY{p}{(}\PY{n}{x}\PY{p}{[}\PY{l+m+mi}{1}\PY{p}{]}\PY{p}{)}\PY{o}{*}\PY{n}{p}\PY{p}{[}\PY{l+m+mi}{1}\PY{p}{]}\PY{o}{*}\PY{n}{sqrt}\PY{p}{(}\PY{l+m+mf}{1.} \PY{o}{/}\PY{p}{(}\PY{n}{x}\PY{p}{[}\PY{l+m+mi}{1}\PY{p}{]}\PY{o}{\PYZca{}}\PY{l+m+mi}{2}\PY{o}{\PYZhy{}}\PY{n}{p}\PY{p}{[}\PY{l+m+mi}{1}\PY{p}{]}\PY{o}{\PYZca{}}\PY{l+m+mi}{2}\PY{p}{)}\PY{p}{)}  \PY{c}{\PYZsh{} contol function}

\PY{c}{\PYZsh{} the Flow function of the control\PYZhy{}toolbox package computes the hamiltonian flow}
\PY{n}{ocp\PYZus{}flow} \PY{o}{=} \PY{n}{Flow}\PY{p}{(}\PY{n}{ocp}\PY{p}{,} \PY{n}{u}\PY{p}{)}  \PY{c}{\PYZsh{} ocp\PYZus{}flow.ode\PYZus{}sol is the result of the solve function from the DifferentailEquations package}

\PY{n}{Int\PYZus{}p0} \PY{o}{=} \PY{o}{\PYZhy{}}\PY{l+m+mf}{0.985}\PY{o}{:}\PY{l+m+mf}{0.25}\PY{o}{:}\PY{l+m+mf}{0.98}   \PY{c}{\PYZsh{} intervalle of p\PYZus{}0}
\PY{n}{Δt} \PY{o}{=} \PY{p}{(}\PY{n}{tf} \PY{o}{\PYZhy{}} \PY{n}{t0}\PY{p}{)}\PY{o}{/}\PY{l+m+mi}{100}          \PY{c}{\PYZsh{} }
\PY{n}{plt\PYZus{}x} \PY{o}{=} \PY{n}{plot}\PY{p}{(}\PY{p}{)}          \PY{c}{\PYZsh{} plot of the state x(t)}
\PY{n}{plt\PYZus{}p} \PY{o}{=} \PY{n}{plot}\PY{p}{(}\PY{p}{)}          \PY{c}{\PYZsh{} plot of the costate p(t)}
\PY{n}{plt\PYZus{}u} \PY{o}{=} \PY{n}{plot}\PY{p}{(}\PY{p}{)}          \PY{c}{\PYZsh{} plot of the control u(t)}
\PY{n}{plt\PYZus{}phase} \PY{o}{=} \PY{n}{plot}\PY{p}{(}\PY{p}{)}      \PY{c}{\PYZsh{} plot (x,p)}

\PY{k}{for} \PY{n}{p0} \PY{k+kp}{in} \PY{n}{Int\PYZus{}p0}        \PY{c}{\PYZsh{} plot for each p\PYZus{}0 in Int\PYZus{}p0 }
    \PY{n}{flow\PYZus{}p0} \PY{o}{=} \PY{n}{ocp\PYZus{}flow}\PY{p}{(}\PY{p}{(}\PY{n}{t0}\PY{p}{,} \PY{n}{tf}\PY{p}{)}\PY{p}{,} \PY{n}{x0}\PY{p}{,} \PY{n}{p0}\PY{p}{,} \PY{n}{reltol} \PY{o}{=} \PY{l+m+mf}{1e\PYZhy{}8}\PY{p}{,} \PY{n}{abstol} \PY{o}{=} \PY{l+m+mf}{1e\PYZhy{}8}\PY{p}{,} \PY{n}{saveat} \PY{o}{=} \PY{n}{Δt}\PY{p}{)}
    \PY{n}{T} \PY{o}{=} \PY{n}{flow\PYZus{}p0}\PY{o}{.}\PY{n}{ode\PYZus{}sol}\PY{o}{.}\PY{n}{t}
    \PY{n}{X} \PY{o}{=} \PY{n}{flow\PYZus{}p0}\PY{o}{.}\PY{n}{ode\PYZus{}sol}\PY{p}{[}\PY{l+m+mi}{1}\PY{p}{,}\PY{o}{:}\PY{p}{]}
    \PY{n}{P} \PY{o}{=} \PY{n}{flow\PYZus{}p0}\PY{o}{.}\PY{n}{ode\PYZus{}sol}\PY{p}{[}\PY{l+m+mi}{2}\PY{p}{,}\PY{o}{:}\PY{p}{]}
    \PY{n}{par} \PY{o}{=} \PY{n}{atanh}\PY{p}{(}\PY{n}{p0}\PY{o}{./}\PY{n}{x0}\PY{p}{)}
    \PY{n}{plot!}\PY{p}{(}\PY{n}{plt\PYZus{}x}\PY{p}{,}\PY{n}{T}\PY{p}{,}\PY{n}{X}\PY{p}{)}
    \PY{n}{plot!}\PY{p}{(}\PY{n}{plt\PYZus{}p}\PY{p}{,}\PY{n}{T}\PY{p}{,}\PY{n}{P}\PY{p}{)}
    \PY{n}{plot!}\PY{p}{(}\PY{n}{plt\PYZus{}u}\PY{p}{,}\PY{n}{T}\PY{p}{,}\PY{n}{u}\PY{o}{.}\PY{p}{(}\PY{n}{X}\PY{p}{,}\PY{n}{P}\PY{p}{)}\PY{p}{)}  
    \PY{n}{plot!}\PY{p}{(}\PY{n}{plt\PYZus{}phase}\PY{p}{,}\PY{n}{X}\PY{p}{,}\PY{n}{P}\PY{p}{)}      
    \PY{c}{\PYZsh{}plot!(plt,flow\PYZus{}p0, control\PYZus{}style=(label=\PYZdq{}u for p\PYZus{}0\PYZdq{},)) \PYZsh{} pb car pas les même ordonnées}
\PY{k}{end}

\PY{c}{\PYZsh{}flow\PYZus{}sol = f((t0, tf), x0, p\PYZus{}sol, saveat = Δt)    }

\PY{c}{\PYZsh{}plot!(plt, flow\PYZus{}sol, control\PYZus{}style=(label=\PYZdq{}u\PYZus{}sol\PYZdq{},))}

\PY{c}{\PYZsh{}plot!(plt[5], ylims=(\PYZhy{}6, 6))}
\PY{n}{plot!}\PY{p}{(}\PY{n}{plt\PYZus{}x}\PY{p}{,}\PY{n}{xlabel} \PY{o}{=} \PY{n}{L}\PY{l+s}{\PYZdq{}}\PY{l+s}{t}\PY{l+s}{\PYZdq{}}\PY{p}{,}\PY{n}{ylabel}\PY{o}{=}\PY{n}{L}\PY{l+s}{\PYZdq{}}\PY{l+s}{x}\PY{l+s}{(}\PY{l+s}{t}\PY{l+s}{,}\PY{l+s}{p}\PY{l+s}{\PYZus{}}\PY{l+s}{0}\PY{l+s}{)}\PY{l+s}{\PYZdq{}}\PY{p}{,}\PY{n}{legend}\PY{o}{=}\PY{k+kc}{false}\PY{p}{,} \PY{n}{ylims}\PY{o}{=}\PY{p}{(}\PY{l+m+mf}{0.}\PY{p}{,}\PY{l+m+mf}{5.}\PY{p}{)}\PY{p}{)}
\PY{n}{plot!}\PY{p}{(}\PY{n}{plt\PYZus{}p}\PY{p}{,}\PY{n}{xlabel} \PY{o}{=} \PY{n}{L}\PY{l+s}{\PYZdq{}}\PY{l+s}{t}\PY{l+s}{\PYZdq{}}\PY{p}{,}\PY{n}{ylabel}\PY{o}{=}\PY{n}{L}\PY{l+s}{\PYZdq{}}\PY{l+s}{p}\PY{l+s}{(}\PY{l+s}{t}\PY{l+s}{,}\PY{l+s}{p}\PY{l+s}{\PYZus{}}\PY{l+s}{0}\PY{l+s}{)}\PY{l+s}{\PYZdq{}}\PY{p}{,}\PY{n}{legend}\PY{o}{=}\PY{k+kc}{false}\PY{p}{,} \PY{n}{ylims}\PY{o}{=}\PY{p}{(}\PY{o}{\PYZhy{}}\PY{l+m+mf}{1.5}\PY{p}{,}\PY{l+m+mf}{5.}\PY{p}{)}\PY{p}{)}
\PY{n}{plot!}\PY{p}{(}\PY{n}{plt\PYZus{}u}\PY{p}{,}\PY{n}{xlabel} \PY{o}{=} \PY{n}{L}\PY{l+s}{\PYZdq{}}\PY{l+s}{t}\PY{l+s}{\PYZdq{}}\PY{p}{,}\PY{n}{ylabel}\PY{o}{=}\PY{n}{L}\PY{l+s}{\PYZdq{}}\PY{l+s}{u}\PY{l+s}{(}\PY{l+s}{t}\PY{l+s}{,}\PY{l+s}{p}\PY{l+s}{\PYZus{}}\PY{l+s}{0}\PY{l+s}{)}\PY{l+s}{\PYZdq{}}\PY{p}{,}\PY{n}{legend}\PY{o}{=}\PY{k+kc}{false}\PY{p}{,} \PY{n}{ylims}\PY{o}{=}\PY{p}{(}\PY{o}{\PYZhy{}}\PY{l+m+mf}{6.}\PY{p}{,}\PY{l+m+mf}{5.}\PY{p}{)}\PY{p}{)}
\PY{n}{plot!}\PY{p}{(}\PY{n}{plt\PYZus{}phase}\PY{p}{,}\PY{n}{xlabel} \PY{o}{=} \PY{n}{L}\PY{l+s}{\PYZdq{}}\PY{l+s}{x}\PY{l+s}{(}\PY{l+s}{t}\PY{l+s}{,}\PY{l+s}{p}\PY{l+s}{\PYZus{}}\PY{l+s}{0}\PY{l+s}{)}\PY{l+s}{\PYZdq{}}\PY{p}{,}\PY{n}{ylabel}\PY{o}{=}\PY{n}{L}\PY{l+s}{\PYZdq{}}\PY{l+s}{p}\PY{l+s}{(}\PY{l+s}{t}\PY{l+s}{,}\PY{l+s}{p}\PY{l+s}{\PYZus{}}\PY{l+s}{0}\PY{l+s}{)}\PY{l+s}{\PYZdq{}}\PY{p}{,}\PY{n}{legend}\PY{o}{=}\PY{k+kc}{false}\PY{p}{,} \PY{n}{xlims}\PY{o}{=}\PY{p}{(}\PY{l+m+mf}{0.}\PY{p}{,}\PY{l+m+mf}{2.}\PY{p}{)}\PY{p}{,} \PY{n}{ylims}\PY{o}{=}\PY{p}{(}\PY{o}{\PYZhy{}}\PY{l+m+mf}{1.}\PY{p}{,}\PY{l+m+mf}{5.}\PY{p}{)}\PY{p}{)}
\PY{n}{plot}\PY{p}{(}\PY{n}{plt\PYZus{}x}\PY{p}{,}\PY{n}{plt\PYZus{}p}\PY{p}{,}\PY{n}{plt\PYZus{}u}\PY{p}{,}\PY{n}{plt\PYZus{}phase}\PY{p}{,}\PY{n}{layout} \PY{o}{=} \PY{p}{(}\PY{l+m+mi}{2}\PY{p}{,}\PY{l+m+mi}{2}\PY{p}{)}\PY{p}{)}
\end{Verbatim}
\end{tcolorbox}

    \begin{center}
    \adjustimage{max size={0.9\linewidth}{0.9\paperheight}}{output_5_0.pdf}
    \end{center}
    { \hspace*{\fill} \\}
    
    \hypertarget{conjugate-points}{%
\subsection{Conjugate points}\label{conjugate-points}}

The time \(\tau\) is said to be conjugate to the the time \(t_0=0\) if
the solution of the Jacobi equation \[
\dot{\delta z}(t) = \dfrac{\partial \vec{H}}{\partial z}(z(t,z_0))\delta z(t)
\] with the initial condition
\(\delta z(0) = \begin{pmatrix}0\\1 \end{pmatrix}\), is vertical at time
\(\tau\), that is \(\delta x(\tau) = 0\).

We first compute by automatic differentiation the flow of the Jacobi
equation with the initial condition
\[\delta z(t,x_0,p_0)=\dfrac{\partial z}{\partial p_0}z(t,x_0,p_0)\]

    \begin{tcolorbox}[breakable, size=fbox, boxrule=1pt, pad at break*=1mm,colback=cellbackground, colframe=cellborder]
\prompt{In}{incolor}{41}{\boxspacing}
\begin{Verbatim}[commandchars=\\\{\}]
\PY{c}{\PYZsh{}}
\PY{c}{\PYZsh{} Conjugate points}
\PY{c}{\PYZsh{}}
\PY{n}{include}\PY{p}{(}\PY{l+s}{\PYZdq{}}\PY{l+s}{.}\PY{l+s}{/}\PY{l+s}{e}\PY{l+s}{x}\PY{l+s}{a}\PY{l+s}{m}\PY{l+s}{p}\PY{l+s}{l}\PY{l+s}{e}\PY{l+s}{1}\PY{l+s}{\PYZhy{}}\PY{l+s}{a}\PY{l+s}{n}\PY{l+s}{a}\PY{l+s}{l}\PY{l+s}{y}\PY{l+s}{t}\PY{l+s}{i}\PY{l+s}{q}\PY{l+s}{u}\PY{l+s}{e}\PY{l+s}{.}\PY{l+s}{j}\PY{l+s}{l}\PY{l+s}{\PYZdq{}}\PY{p}{)}
\PY{l+s}{\PYZdq{}\PYZdq{}\PYZdq{}}
\PY{l+s}{ }\PY{l+s}{ }\PY{l+s}{ }\PY{l+s}{ }\PY{l+s}{C}\PY{l+s}{o}\PY{l+s}{m}\PY{l+s}{p}\PY{l+s}{u}\PY{l+s}{t}\PY{l+s}{e}\PY{l+s}{ }\PY{l+s}{t}\PY{l+s}{h}\PY{l+s}{e}\PY{l+s}{ }\PY{l+s}{f}\PY{l+s}{l}\PY{l+s}{o}\PY{l+s}{w}\PY{l+s}{ }\PY{l+s}{o}\PY{l+s}{f}\PY{l+s}{ }\PY{l+s}{t}\PY{l+s}{h}\PY{l+s}{e}\PY{l+s}{ }\PY{l+s}{J}\PY{l+s}{a}\PY{l+s}{c}\PY{l+s}{o}\PY{l+s}{b}\PY{l+s}{i}\PY{l+s}{ }\PY{l+s}{e}\PY{l+s}{q}\PY{l+s}{u}\PY{l+s}{a}\PY{l+s}{t}\PY{l+s}{i}\PY{l+s}{o}\PY{l+s}{n}\PY{l+s}{ }\PY{l+s}{f}\PY{l+s}{o}\PY{l+s}{r}\PY{l+s}{ }\PY{l+s}{t}\PY{l+s}{h}\PY{l+s}{e}\PY{l+s}{ }\PY{l+s}{i}\PY{l+s}{n}\PY{l+s}{i}\PY{l+s}{t}\PY{l+s}{i}\PY{l+s}{a}\PY{l+s}{l}\PY{l+s}{ }\PY{l+s}{c}\PY{l+s}{o}\PY{l+s}{n}\PY{l+s}{d}\PY{l+s}{i}\PY{l+s}{t}\PY{l+s}{i}\PY{l+s}{o}\PY{l+s}{n}\PY{l+s}{ }\PY{l+s}{δ}\PY{l+s}{z}\PY{l+s}{(}\PY{l+s}{0}\PY{l+s}{)}\PY{l+s}{ }\PY{l+s}{=}\PY{l+s}{ }\PY{l+s}{(}\PY{l+s}{0}\PY{l+s}{,}\PY{l+s}{1}\PY{l+s}{)}
\PY{l+s}{ }\PY{l+s}{ }\PY{l+s}{ }\PY{l+s}{ }\PY{l+s}{s}\PY{l+s}{o}\PY{l+s}{l}\PY{l+s}{ }\PY{l+s}{=}\PY{l+s}{ }\PY{l+s}{f}\PY{l+s}{l}\PY{l+s}{o}\PY{l+s}{w}\PY{l+s}{\PYZus{}}\PY{l+s}{j}\PY{l+s}{a}\PY{l+s}{c}\PY{l+s}{o}\PY{l+s}{b}\PY{l+s}{i}\PY{l+s}{(}\PY{l+s}{T}\PY{l+s}{,}\PY{l+s}{x}\PY{l+s}{0}\PY{l+s}{,}\PY{l+s}{p}\PY{l+s}{0}\PY{l+s}{)}
\PY{l+s}{ }\PY{l+s}{ }\PY{l+s}{ }\PY{l+s}{ }\PY{l+s}{i}\PY{l+s}{n}\PY{l+s}{p}\PY{l+s}{u}\PY{l+s}{t}\PY{l+s}{ }\PY{l+s}{:}\PY{l+s}{ }
\PY{l+s}{ }\PY{l+s}{ }\PY{l+s}{ }\PY{l+s}{ }\PY{l+s}{T}\PY{l+s}{ }\PY{l+s}{:}\PY{l+s}{ }\PY{l+s}{t}\PY{l+s}{i}\PY{l+s}{m}\PY{l+s}{e}\PY{l+s}{ }\PY{l+s}{w}\PY{l+s}{h}\PY{l+s}{e}\PY{l+s}{r}\PY{l+s}{e}\PY{l+s}{ }\PY{l+s}{w}\PY{l+s}{e}\PY{l+s}{ }\PY{l+s}{w}\PY{l+s}{a}\PY{l+s}{n}\PY{l+s}{t}\PY{l+s}{ }\PY{l+s}{δ}\PY{l+s}{z}\PY{l+s}{(}\PY{l+s}{t}\PY{l+s}{)}\PY{l+s}{ }
\PY{l+s}{ }\PY{l+s}{ }\PY{l+s}{ }\PY{l+s}{ }\PY{l+s}{ }\PY{l+s}{ }\PY{l+s}{ }\PY{l+s}{ }\PY{l+s}{l}\PY{l+s}{i}\PY{l+s}{s}\PY{l+s}{t}\PY{l+s}{ }\PY{l+s}{o}\PY{l+s}{r}\PY{l+s}{ }\PY{l+s}{a}\PY{l+s}{r}\PY{l+s}{r}\PY{l+s}{a}\PY{l+s}{y}\PY{l+s}{ }\PY{l+s}{(}\PY{l+s}{t}\PY{l+s}{0}\PY{l+s}{,}\PY{l+s}{t}\PY{l+s}{1}\PY{l+s}{,}\PY{l+s}{.}\PY{l+s}{.}\PY{l+s}{.}\PY{l+s}{,}\PY{l+s}{t}\PY{l+s}{N}\PY{l+s}{)}
\PY{l+s}{ }\PY{l+s}{ }\PY{l+s}{ }\PY{l+s}{ }\PY{l+s}{x}\PY{l+s}{0}\PY{l+s}{ }\PY{l+s}{:}\PY{l+s}{ }\PY{l+s}{i}\PY{l+s}{n}\PY{l+s}{i}\PY{l+s}{t}\PY{l+s}{i}\PY{l+s}{a}\PY{l+s}{l}\PY{l+s}{ }\PY{l+s}{s}\PY{l+s}{t}\PY{l+s}{a}\PY{l+s}{t}\PY{l+s}{e}
\PY{l+s}{ }\PY{l+s}{ }\PY{l+s}{ }\PY{l+s}{ }\PY{l+s}{ }\PY{l+s}{ }\PY{l+s}{ }\PY{l+s}{ }\PY{l+s}{ }\PY{l+s}{R}\PY{l+s}{e}\PY{l+s}{a}\PY{l+s}{l}\PY{l+s}{(}\PY{l+s}{n}\PY{l+s}{)}\PY{l+s}{,}\PY{l+s}{ }\PY{l+s}{h}\PY{l+s}{e}\PY{l+s}{r}\PY{l+s}{e}\PY{l+s}{ }\PY{l+s}{a}\PY{l+s}{ }\PY{l+s}{r}\PY{l+s}{e}\PY{l+s}{a}\PY{l+s}{l}
\PY{l+s}{ }\PY{l+s}{ }\PY{l+s}{ }\PY{l+s}{ }\PY{l+s}{p}\PY{l+s}{0}\PY{l+s}{ }\PY{l+s}{:}\PY{l+s}{ }\PY{l+s}{i}\PY{l+s}{n}\PY{l+s}{i}\PY{l+s}{t}\PY{l+s}{i}\PY{l+s}{a}\PY{l+s}{l}\PY{l+s}{ }\PY{l+s}{c}\PY{l+s}{o}\PY{l+s}{s}\PY{l+s}{t}\PY{l+s}{a}\PY{l+s}{t}\PY{l+s}{e}
\PY{l+s}{ }\PY{l+s}{ }\PY{l+s}{ }\PY{l+s}{ }\PY{l+s}{ }\PY{l+s}{ }\PY{l+s}{ }\PY{l+s}{ }\PY{l+s}{ }\PY{l+s}{R}\PY{l+s}{e}\PY{l+s}{a}\PY{l+s}{l}\PY{l+s}{(}\PY{l+s}{n}\PY{l+s}{)}\PY{l+s}{,}\PY{l+s}{ }\PY{l+s}{h}\PY{l+s}{e}\PY{l+s}{r}\PY{l+s}{e}\PY{l+s}{ }\PY{l+s}{a}\PY{l+s}{ }\PY{l+s}{r}\PY{l+s}{e}\PY{l+s}{a}\PY{l+s}{l}
\PY{l+s}{ }\PY{l+s}{ }\PY{l+s}{ }\PY{l+s}{ }\PY{l+s}{o}\PY{l+s}{u}\PY{l+s}{t}\PY{l+s}{p}\PY{l+s}{u}\PY{l+s}{t}\PY{l+s}{ }\PY{l+s}{:}\PY{l+s}{ }
\PY{l+s}{ }\PY{l+s}{ }\PY{l+s}{ }\PY{l+s}{ }\PY{l+s}{s}\PY{l+s}{o}\PY{l+s}{l}\PY{l+s}{ }\PY{l+s}{:}\PY{l+s}{ }\PY{l+s}{(}\PY{l+s}{δ}\PY{l+s}{z}\PY{l+s}{(}\PY{l+s}{t}\PY{l+s}{0}\PY{l+s}{)}\PY{l+s}{,}\PY{l+s}{δ}\PY{l+s}{z}\PY{l+s}{(}\PY{l+s}{t}\PY{l+s}{1}\PY{l+s}{)}\PY{l+s}{,}\PY{l+s}{.}\PY{l+s}{.}\PY{l+s}{.}\PY{l+s}{,}\PY{l+s}{δ}\PY{l+s}{z}\PY{l+s}{(}\PY{l+s}{t}\PY{l+s}{N}\PY{l+s}{)}\PY{l+s}{)}
\PY{l+s}{ }\PY{l+s}{ }\PY{l+s}{ }\PY{l+s}{ }\PY{l+s}{ }\PY{l+s}{ }\PY{l+s}{ }\PY{l+s}{ }\PY{l+s}{ }\PY{l+s}{ }\PY{l+s}{V}\PY{l+s}{e}\PY{l+s}{c}\PY{l+s}{t}\PY{l+s}{o}\PY{l+s}{r}\PY{l+s}{ }\PY{l+s}{o}\PY{l+s}{f}\PY{l+s}{ }\PY{l+s}{v}\PY{l+s}{e}\PY{l+s}{c}\PY{l+s}{t}\PY{l+s}{o}\PY{l+s}{r}\PY{l+s}{ }\PY{l+s}{o}\PY{l+s}{f}\PY{l+s}{ }\PY{l+s}{d}\PY{l+s}{i}\PY{l+s}{m}\PY{l+s}{e}\PY{l+s}{n}\PY{l+s}{s}\PY{l+s}{i}\PY{l+s}{o}\PY{l+s}{n}\PY{l+s}{ }\PY{l+s}{2}\PY{l+s}{n}\PY{l+s}{,}\PY{l+s}{ }\PY{l+s}{t}\PY{l+s}{h}\PY{l+s}{e}\PY{l+s}{ }\PY{l+s}{t}\PY{l+s}{y}\PY{l+s}{p}\PY{l+s}{e}\PY{l+s}{ }\PY{l+s}{o}\PY{l+s}{f}\PY{l+s}{ }\PY{l+s}{t}\PY{l+s}{h}\PY{l+s}{e}\PY{l+s}{ }\PY{l+s}{e}\PY{l+s}{l}\PY{l+s}{e}\PY{l+s}{m}\PY{l+s}{e}\PY{l+s}{n}\PY{l+s}{t}\PY{l+s}{ }\PY{l+s}{a}\PY{l+s}{r}\PY{l+s}{e}\PY{l+s}{ }\PY{l+s}{t}\PY{l+s}{h}\PY{l+s}{e}\PY{l+s}{ }\PY{l+s}{s}\PY{l+s}{a}\PY{l+s}{m}\PY{l+s}{e}\PY{l+s}{ }\PY{l+s}{a}\PY{l+s}{s}\PY{l+s}{ }\PY{l+s}{t}\PY{l+s}{h}\PY{l+s}{e}\PY{l+s}{ }\PY{l+s}{t}\PY{l+s}{y}\PY{l+s}{p}\PY{l+s}{e}\PY{l+s}{ }\PY{l+s}{o}\PY{l+s}{f}\PY{l+s}{ }\PY{l+s}{p}\PY{l+s}{0}\PY{l+s}{ }\PY{l+s}{b}\PY{l+s}{e}\PY{l+s}{c}\PY{l+s}{a}\PY{l+s}{u}\PY{l+s}{s}\PY{l+s}{e}\PY{l+s}{ }\PY{l+s}{w}\PY{l+s}{e}\PY{l+s}{ }\PY{l+s}{u}\PY{l+s}{s}\PY{l+s}{e}\PY{l+s}{ }\PY{l+s}{a}\PY{l+s}{f}\PY{l+s}{t}\PY{l+s}{e}\PY{l+s}{r}\PY{l+s}{ }\PY{l+s}{t}\PY{l+s}{h}\PY{l+s}{e}\PY{l+s}{ }\PY{l+s}{a}\PY{l+s}{u}\PY{l+s}{t}\PY{l+s}{o}\PY{l+s}{m}\PY{l+s}{a}\PY{l+s}{t}\PY{l+s}{i}\PY{l+s}{c}\PY{l+s}{ }\PY{l+s}{d}\PY{l+s}{i}\PY{l+s}{f}\PY{l+s}{f}\PY{l+s}{e}\PY{l+s}{r}\PY{l+s}{e}\PY{l+s}{n}\PY{l+s}{t}\PY{l+s}{i}\PY{l+s}{a}\PY{l+s}{t}\PY{l+s}{i}\PY{l+s}{o}\PY{l+s}{n}\PY{l+s}{.}
\PY{l+s}{ }\PY{l+s}{ }\PY{l+s}{ }\PY{l+s}{ }\PY{l+s}{ }\PY{l+s}{ }\PY{l+s}{ }\PY{l+s}{ }\PY{l+s}{ }\PY{l+s}{ }\PY{l+s}{T}\PY{l+s}{o}\PY{l+s}{ }\PY{l+s}{m}\PY{l+s}{o}\PY{l+s}{d}\PY{l+s}{i}\PY{l+s}{f}\PY{l+s}{i}\PY{l+s}{e}\PY{l+s}{ }\PY{l+s}{i}\PY{l+s}{f}\PY{l+s}{ }\PY{l+s}{n}\PY{l+s}{ }\PY{l+s}{\PYZgt{}}\PY{l+s}{ }\PY{l+s}{1}

\PY{l+s}{\PYZdq{}\PYZdq{}\PYZdq{}}
\PY{k}{function} \PY{n}{flow\PYZus{}jacobi}\PY{p}{(}\PY{n}{T}\PY{p}{,}\PY{n}{x0}\PY{p}{,}\PY{n}{p0}\PY{p}{)}
    \PY{n}{n} \PY{o}{=} \PY{n}{length}\PY{p}{(}\PY{n}{x0}\PY{p}{)}
    \PY{n}{t0} \PY{o}{=} \PY{n}{T}\PY{p}{[}\PY{l+m+mi}{1}\PY{p}{]}
    \PY{n}{nb\PYZus{}t} \PY{o}{=} \PY{n}{length}\PY{p}{(}\PY{n}{T}\PY{p}{)}
    \PY{c}{\PYZsh{} type because we use ForwardDiff}
    \PY{n}{sol} \PY{o}{=} \PY{p}{[}\PY{k+kt}{Vector}\PY{p}{\PYZob{}}\PY{n}{typeof}\PY{p}{(}\PY{n}{p0}\PY{p}{[}\PY{l+m+mi}{1}\PY{p}{]}\PY{p}{)}\PY{p}{\PYZcb{}}\PY{p}{(}\PY{n}{undef}\PY{p}{,}\PY{l+m+mi}{2}\PY{n}{n}\PY{p}{)} \PY{k}{for} \PY{n}{\PYZus{}} \PY{k+kp}{in} \PY{l+m+mi}{1}\PY{o}{:}\PY{n}{nb\PYZus{}t}\PY{p}{]} \PY{c}{\PYZsh{} Vector \PYZhy{}\PYZgt{} Matrix if n \PYZgt{}  1}
                                                         \PY{c}{\PYZsh{} typeof(p0) \PYZhy{}\PYZgt{} typeof(p0[1]), type of the element of p0}
    \PY{k}{for} \PY{n}{i} \PY{k+kp}{in} \PY{l+m+mi}{1}\PY{o}{:}\PY{n}{nb\PYZus{}t}
        \PY{n}{t} \PY{o}{=} \PY{n}{T}\PY{p}{[}\PY{n}{i}\PY{p}{]}
        \PY{c}{\PYZsh{} flow\PYZus{}p0 compute de flow at t oh the hamiltonian system}
        \PY{n}{flow\PYZus{}p0}\PY{p}{(}\PY{n}{p\PYZus{}0}\PY{p}{)} \PY{o}{=} \PY{n}{ocp\PYZus{}flow}\PY{p}{(}\PY{p}{(}\PY{n}{t0}\PY{p}{,} \PY{n}{t}\PY{p}{)}\PY{p}{,} \PY{n}{x0}\PY{p}{,} \PY{n}{p\PYZus{}0}\PY{p}{,} \PY{n}{reltol} \PY{o}{=} \PY{l+m+mf}{1e\PYZhy{}10}\PY{p}{,} \PY{n}{abstol} \PY{o}{=} \PY{l+m+mf}{1e\PYZhy{}10}\PY{p}{)}\PY{o}{.}\PY{n}{ode\PYZus{}sol}\PY{o}{.}\PY{n}{u}\PY{p}{[}\PY{k}{end}\PY{p}{]}
        \PY{c}{\PYZsh{} temp is the solution at t of the derivative with respect to p0 of the flow}
        \PY{n}{temp} \PY{o}{=} \PY{n}{ForwardDiff}\PY{o}{.}\PY{n}{derivative}\PY{p}{(}\PY{n}{flow\PYZus{}p0}\PY{p}{,} \PY{n}{p0}\PY{p}{)}  \PY{c}{\PYZsh{} p0 is a real}
        \PY{c}{\PYZsh{}sol[i] = det(temp[1:n,1:n])}
        \PY{n}{sol}\PY{p}{[}\PY{n}{i}\PY{p}{]} \PY{o}{=} \PY{n}{temp}
    \PY{k}{end}
    \PY{k}{return} \PY{n}{sol}
\PY{k}{end}

\PY{n}{p0} \PY{o}{=} \PY{o}{\PYZhy{}}\PY{l+m+mf}{0.985}
\PY{n}{t0tf} \PY{o}{=} \PY{p}{(}\PY{l+m+mf}{0.}\PY{p}{,} \PY{l+m+mf}{2.}\PY{p}{)}
\PY{c}{\PYZsh{}println(flow\PYZus{}jacobi(t0tf,x0,p0)[end] \PYZhy{} flow\PYZus{}jacobi\PYZus{}ana(t0tf,x0,p0).u[end])}
\PY{c}{\PYZsh{}@test isapprox(flow\PYZus{}jacobi(t0tf,x0,p0)[end] ,flow\PYZus{}jacobi\PYZus{}ana(t0tf,x0,p0).u[end]; rtol = 1.e\PYZhy{}3)}

\PY{n}{Int\PYZus{}t0tf} \PY{o}{=} \PY{n}{range}\PY{p}{(}\PY{n}{t0}\PY{p}{,}\PY{n}{stop}\PY{o}{=}\PY{n}{tf}\PY{p}{,}\PY{n}{length}\PY{o}{=}\PY{l+m+mi}{100}\PY{p}{)}
\PY{n}{sol} \PY{o}{=} \PY{n}{flow\PYZus{}jacobi}\PY{p}{(}\PY{n}{Int\PYZus{}t0tf}\PY{p}{,}\PY{n}{x0}\PY{p}{,}\PY{n}{p0}\PY{p}{)}
\PY{n}{plt\PYZus{}conj1} \PY{o}{=} \PY{n}{plot}\PY{p}{(}\PY{p}{)}
\PY{n}{plot!}\PY{p}{(}\PY{n}{plt\PYZus{}conj1}\PY{p}{,}\PY{n}{Int\PYZus{}t0tf}\PY{p}{,}\PY{p}{[}\PY{n}{sol}\PY{p}{[}\PY{n}{i}\PY{p}{]}\PY{p}{[}\PY{l+m+mi}{1}\PY{p}{]} \PY{k}{for} \PY{n}{i} \PY{k+kp}{in} \PY{l+m+mi}{1}\PY{o}{:}\PY{n}{length}\PY{p}{(}\PY{n}{sol}\PY{p}{)}\PY{p}{]}\PY{p}{)}  \PY{c}{\PYZsh{} as n=1 the det is the number}
\PY{n}{plot!}\PY{p}{(}\PY{n}{plt\PYZus{}conj1}\PY{p}{,}\PY{n}{xlabel} \PY{o}{=} \PY{n}{L}\PY{l+s}{\PYZdq{}}\PY{l+s}{t}\PY{l+s}{\PYZdq{}}\PY{p}{,}\PY{n}{ylabel}\PY{o}{=}\PY{n}{L}\PY{l+s}{\PYZdq{}}\PY{l+s}{\PYZbs{}}\PY{l+s}{d}\PY{l+s}{e}\PY{l+s}{l}\PY{l+s}{t}\PY{l+s}{a}\PY{l+s}{ }\PY{l+s}{x}\PY{l+s}{(}\PY{l+s}{t}\PY{l+s}{,}\PY{l+s}{p}\PY{l+s}{\PYZus{}}\PY{l+s}{0}\PY{l+s}{)}\PY{l+s}{\PYZdq{}}\PY{p}{,}\PY{n}{legend}\PY{o}{=}\PY{k+kc}{false}\PY{p}{,} \PY{n}{ylims}\PY{o}{=}\PY{p}{(}\PY{o}{\PYZhy{}}\PY{l+m+mf}{10.}\PY{p}{,}\PY{l+m+mf}{10.}\PY{p}{)}\PY{p}{)}
\PY{n}{plot}\PY{p}{(}\PY{n}{plt\PYZus{}conj1}\PY{p}{)}
\end{Verbatim}
\end{tcolorbox}

    \begin{center}
    \adjustimage{max size={0.9\linewidth}{0.9\paperheight}}{output_7_0.pdf}
    \end{center}
    { \hspace*{\fill} \\}
    
    Then we numerically compute the conjugate point by solving
\(\delta x(t)=\delta z(t,x_0,p_0)_1=0\), for \(x_0=1.\) and
\(p_0=0.985\).

    \begin{tcolorbox}[breakable, size=fbox, boxrule=1pt, pad at break*=1mm,colback=cellbackground, colframe=cellborder]
\prompt{In}{incolor}{42}{\boxspacing}
\begin{Verbatim}[commandchars=\\\{\}]
\PY{c}{\PYZsh{} compute the fisrt conjugate point}
\PY{k}{function} \PY{n}{F}\PY{p}{(}\PY{n}{t0}\PY{p}{,}\PY{n}{τ}\PY{p}{,}\PY{n}{x0}\PY{p}{,}\PY{n}{p0}\PY{p}{)}
    \PY{n}{tspan} \PY{o}{=} \PY{p}{(}\PY{n}{t0}\PY{p}{,}\PY{n}{τ}\PY{p}{)}
\PY{c}{\PYZsh{}    return [flow\PYZus{}jacobi(tspan,x₀,p0)(τ)[1]]}
    \PY{k}{return} \PY{n}{flow\PYZus{}jacobi}\PY{p}{(}\PY{n}{tspan}\PY{p}{,}\PY{n}{x0}\PY{p}{,}\PY{n}{p0}\PY{p}{)}\PY{p}{[}\PY{k}{end}\PY{p}{]}
\PY{k}{end}

\PY{c}{\PYZsh{} compute τ0}
\PY{n}{δx}\PY{p}{(}\PY{n}{τ}\PY{p}{)} \PY{o}{=} \PY{n}{F}\PY{p}{(}\PY{n}{t0}\PY{p}{,}\PY{n}{τ}\PY{p}{,}\PY{n}{x0}\PY{p}{,}\PY{n}{p0}\PY{p}{)}\PY{p}{[}\PY{l+m+mi}{1}\PY{p}{]}

\PY{c}{\PYZsh{}sol = nlsolve(δx,[0.5])}
\PY{k}{using} \PY{n}{Roots}
\PY{c}{\PYZsh{}τ0 = sol.zero}
\PY{n}{τ0} \PY{o}{=} \PY{n}{find\PYZus{}zero}\PY{p}{(}\PY{n}{δx}\PY{p}{,} \PY{p}{(}\PY{l+m+mf}{0.4}\PY{p}{,} \PY{l+m+mf}{0.6}\PY{p}{)}\PY{p}{)}
\PY{n}{println}\PY{p}{(}\PY{l+s}{\PYZdq{}}\PY{l+s}{F}\PY{l+s}{o}\PY{l+s}{r}\PY{l+s}{ }\PY{l+s}{p}\PY{l+s}{0}\PY{l+s}{ }\PY{l+s}{=}\PY{l+s}{ }\PY{l+s}{\PYZdq{}}\PY{p}{,} \PY{n}{p0}\PY{p}{,} \PY{l+s}{\PYZdq{}}\PY{l+s}{ }\PY{l+s}{t}\PY{l+s}{a}\PY{l+s}{u}\PY{l+s}{\PYZus{}}\PY{l+s}{0}\PY{l+s}{ }\PY{l+s}{=}\PY{l+s}{ }\PY{l+s}{\PYZdq{}}\PY{p}{,} \PY{n}{τ0}\PY{p}{)}

\PY{n}{plot!}\PY{p}{(}\PY{n}{plt\PYZus{}conj1}\PY{p}{,}\PY{p}{[}\PY{n}{τ0}\PY{p}{]}\PY{p}{,}\PY{p}{[}\PY{n}{flow\PYZus{}jacobi}\PY{p}{(}\PY{p}{(}\PY{n}{t0}\PY{p}{,}\PY{n}{τ0}\PY{p}{)}\PY{p}{,}\PY{n}{x0}\PY{p}{,}\PY{n}{p0}\PY{p}{)}\PY{p}{[}\PY{k}{end}\PY{p}{]}\PY{p}{]}\PY{p}{,}\PY{n}{seriestype}\PY{o}{=}\PY{o}{:}\PY{n}{scatter}\PY{p}{)}
\end{Verbatim}
\end{tcolorbox}

    \begin{Verbatim}[commandchars=\\\{\}]
For p0 = -0.985 tau\_0 = 0.51728931341542
    \end{Verbatim}

    \begin{center}
    \adjustimage{max size={0.9\linewidth}{0.9\paperheight}}{output_9_1.pdf}
    \end{center}
    { \hspace*{\fill} \\}
    
    \hypertarget{compute-the-conjugate-loci}{%
\subsection{Compute the conjugate
loci}\label{compute-the-conjugate-loci}}

We compute cojugate loci by path following algorithm

We define \(F(\tau,p_0) = \delta x(\tau,p_0)\) and we suppose that
\(\dfrac{\partial F}{\partial\tau}(\tau,p_0)\) is inversible, then by
the implicit function theorem the conjugate time is a function of
\(p_0\). So, as here \(p_0\in\R\), we can compute them by solving the
initial value problem \[(IVP_{conj.points})\left\{
\begin{array}{l}
\dot{\tau} = -\dfrac{\partial F}{\partial\tau}(\tau,p_0)^{-1}\dfrac{\partial F}{\partial p_0}(\tau,p_0)\\
\tau(p_0) = \tau_0.
\end{array}
    \right.\]

\hypertarget{remark}{%
\subsubsection{Remark}\label{remark}}

The derivative
\(\dfrac{\partial F}{\partial\tau}(\tau,p_0) = \dfrac{\partial \delta x}{\partial\tau}(\tau,p_0)\)
is equal to the first component of the second member of the Jacobi
equation \(\dfrac{\partial \vec{H}}{\partial z}(z(t,z_0))\delta z(t)\).

    \begin{tcolorbox}[breakable, size=fbox, boxrule=1pt, pad at break*=1mm,colback=cellbackground, colframe=cellborder]
\prompt{In}{incolor}{45}{\boxspacing}
\begin{Verbatim}[commandchars=\\\{\}]
\PY{c}{\PYZsh{} Test of Hvec }
    \PY{n}{Hvec!} \PY{o}{=} \PY{n}{ocp\PYZus{}flow}\PY{o}{.}\PY{n}{rhs!} \PY{c}{\PYZsh{} }
    \PY{c}{\PYZsh{}z0 = [1.,\PYZhy{}0.1]}
    \PY{n}{par} \PY{o}{=} \PY{l+m+mf}{0.}
    \PY{n}{z0} \PY{o}{=} \PY{p}{[}\PY{n}{x0} \PY{p}{,} \PY{n}{p0}\PY{p}{]}
    \PY{n}{zpoint} \PY{o}{=} \PY{n}{similar}\PY{p}{(}\PY{n}{z0}\PY{p}{)}
    \PY{n}{Hvec!}\PY{p}{(}\PY{n}{zpoint}\PY{p}{,}\PY{n}{z0}\PY{p}{,}\PY{n}{par}\PY{p}{,}\PY{n}{t0}\PY{p}{)}
    \PY{n}{atol} \PY{o}{=} \PY{l+m+mf}{1.e\PYZhy{}12}
    \PY{n+nd}{@test}  \PY{n}{zpoint} \PY{n}{≈} \PY{n}{H\PYZus{}vec}\PY{p}{(}\PY{n}{z0}\PY{p}{)} \PY{n}{atol} \PY{o}{=} \PY{n}{atol}   \PY{c}{\PYZsh{} test for the Hvec! function}

\PY{c}{\PYZsh{} }


\PY{c}{\PYZsh{}conjugates points by path following}
\PY{l+s}{\PYZdq{}\PYZdq{}\PYZdq{}}
\PY{l+s}{ }\PY{l+s}{ }\PY{l+s}{ }\PY{l+s}{ }\PY{l+s}{C}\PY{l+s}{o}\PY{l+s}{m}\PY{l+s}{p}\PY{l+s}{u}\PY{l+s}{t}\PY{l+s}{e}\PY{l+s}{ }\PY{l+s}{t}\PY{l+s}{h}\PY{l+s}{e}\PY{l+s}{ }\PY{l+s}{r}\PY{l+s}{i}\PY{l+s}{g}\PY{l+s}{h}\PY{l+s}{t}\PY{l+s}{ }\PY{l+s}{h}\PY{l+s}{a}\PY{l+s}{n}\PY{l+s}{f}\PY{l+s}{ }\PY{l+s}{s}\PY{l+s}{i}\PY{l+s}{d}\PY{l+s}{e}\PY{l+s}{ }\PY{l+s}{o}\PY{l+s}{f}\PY{l+s}{ }\PY{l+s}{t}\PY{l+s}{h}\PY{l+s}{e}\PY{l+s}{ }\PY{l+s}{c}\PY{l+s}{o}\PY{l+s}{n}\PY{l+s}{j}\PY{l+s}{.}\PY{l+s}{p}\PY{l+s}{o}\PY{l+s}{i}\PY{l+s}{n}\PY{l+s}{t}\PY{l+s}{s}\PY{l+s}{ }\PY{l+s}{I}\PY{l+s}{V}\PY{l+s}{P}\PY{l+s}{ }\PY{l+s}{e}\PY{l+s}{q}\PY{l+s}{u}\PY{l+s}{a}\PY{l+s}{t}\PY{l+s}{i}\PY{l+s}{o}\PY{l+s}{n}
\PY{l+s}{ }\PY{l+s}{ }\PY{l+s}{ }\PY{l+s}{ }\PY{l+s}{t}\PY{l+s}{a}\PY{l+s}{u}\PY{l+s}{\PYZus{}}\PY{l+s}{p}\PY{l+s}{o}\PY{l+s}{i}\PY{l+s}{n}\PY{l+s}{t}\PY{l+s}{ }\PY{l+s}{=}\PY{l+s}{ }\PY{l+s}{r}\PY{l+s}{h}\PY{l+s}{s}\PY{l+s}{\PYZus{}}\PY{l+s}{p}\PY{l+s}{a}\PY{l+s}{t}\PY{l+s}{h}\PY{l+s}{(}\PY{l+s}{t}\PY{l+s}{a}\PY{l+s}{u}\PY{l+s}{ }\PY{l+s}{,}\PY{l+s}{ }\PY{l+s}{p}\PY{l+s}{a}\PY{l+s}{r}\PY{l+s}{,}\PY{l+s}{ }\PY{l+s}{p}\PY{l+s}{0}\PY{l+s}{)}
\PY{l+s}{ }\PY{l+s}{ }\PY{l+s}{ }\PY{l+s}{ }\PY{l+s}{F}\PY{l+s}{o}\PY{l+s}{r}\PY{l+s}{ }\PY{l+s}{t}\PY{l+s}{h}\PY{l+s}{e}\PY{l+s}{ }\PY{l+s}{s}\PY{l+s}{t}\PY{l+s}{r}\PY{l+s}{u}\PY{l+s}{c}\PY{l+s}{t}\PY{l+s}{u}\PY{l+s}{r}\PY{l+s}{e}\PY{l+s}{ }\PY{l+s}{o}\PY{l+s}{f}\PY{l+s}{ }\PY{l+s}{t}\PY{l+s}{h}\PY{l+s}{e}\PY{l+s}{ }\PY{l+s}{r}\PY{l+s}{h}\PY{l+s}{s}\PY{l+s}{\PYZus{}}\PY{l+s}{p}\PY{l+s}{a}\PY{l+s}{t}\PY{l+s}{h}\PY{l+s}{ }\PY{l+s}{s}\PY{l+s}{e}\PY{l+s}{e}\PY{l+s}{ }\PY{l+s}{D}\PY{l+s}{i}\PY{l+s}{f}\PY{l+s}{f}\PY{l+s}{e}\PY{l+s}{r}\PY{l+s}{e}\PY{l+s}{n}\PY{l+s}{t}\PY{l+s}{i}\PY{l+s}{a}\PY{l+s}{l}\PY{l+s}{E}\PY{l+s}{q}\PY{l+s}{u}\PY{l+s}{a}\PY{l+s}{i}\PY{l+s}{o}\PY{l+s}{n}\PY{l+s}{s}\PY{l+s}{ }\PY{l+s}{ }\PY{l+s}{p}\PY{l+s}{a}\PY{l+s}{c}\PY{l+s}{k}\PY{l+s}{a}\PY{l+s}{g}\PY{l+s}{e}
\PY{l+s}{\PYZdq{}\PYZdq{}\PYZdq{}}
\PY{k}{function} \PY{n}{rhs\PYZus{}path}\PY{p}{(}\PY{n}{tau} \PY{p}{,} \PY{n}{par}\PY{p}{,} \PY{n}{p0}\PY{p}{)}
    \PY{n}{n} \PY{o}{=} \PY{n}{length}\PY{p}{(}\PY{n}{p0}\PY{p}{)}
    \PY{n}{τ} \PY{o}{=} \PY{n}{tau}\PY{p}{[}\PY{l+m+mi}{1}\PY{p}{]}     \PY{c}{\PYZsh{} tau is a vector}
    \PY{n}{z0} \PY{o}{=} \PY{p}{[}\PY{n}{x0}\PY{p}{,}\PY{n}{p0}\PY{p}{]}
    \PY{n}{Hvec!}\PY{p}{(}\PY{n}{zpoint}\PY{p}{,}\PY{n}{z0}\PY{p}{)} \PY{o}{=} \PY{n}{ocp\PYZus{}flow}\PY{o}{.}\PY{n}{rhs!}\PY{p}{(}\PY{n}{zpoint}\PY{p}{,}\PY{n}{z0}\PY{p}{,}\PY{n}{par}\PY{p}{,}\PY{n}{τ}\PY{p}{)}
    \PY{n}{Hvec!}\PY{p}{(}\PY{n}{zpoint}\PY{p}{,}\PY{n}{z0}\PY{p}{)}     \PY{c}{\PYZsh{} zpoint is the second member of the Hamiltonian flow}
    \PY{n}{dHvec} \PY{o}{=} \PY{k+kt}{Matrix}\PY{p}{\PYZob{}}\PY{n}{typeof}\PY{p}{(}\PY{n}{p0}\PY{p}{[}\PY{l+m+mi}{1}\PY{p}{]}\PY{p}{)}\PY{p}{\PYZcb{}}\PY{p}{(}\PY{n}{undef}\PY{p}{,}\PY{l+m+mi}{2}\PY{n}{n}\PY{p}{,}\PY{l+m+mi}{2}\PY{n}{n}\PY{p}{)} \PY{c}{\PYZsh{}zeros(2*n,2*n)}
    \PY{n}{z} \PY{o}{=} \PY{n}{ocp\PYZus{}flow}\PY{p}{(}\PY{p}{(}\PY{n}{t0}\PY{p}{,} \PY{n}{τ}\PY{p}{)}\PY{p}{,} \PY{n}{x0}\PY{p}{,} \PY{n}{p0}\PY{p}{,} \PY{n}{reltol} \PY{o}{=} \PY{l+m+mf}{1e\PYZhy{}8}\PY{p}{,} \PY{n}{abstol} \PY{o}{=} \PY{l+m+mf}{1e\PYZhy{}8}\PY{p}{)}\PY{o}{.}\PY{n}{ode\PYZus{}sol}\PY{p}{(}\PY{n}{τ}\PY{p}{)} \PY{c}{\PYZsh{} compute z(τ)}
    \PY{c}{\PYZsh{} Compute matrix \PYZbs{}dfrac\PYZob{}\PYZbs{}partial \PYZbs{}vec\PYZob{}H\PYZcb{}\PYZcb{}\PYZob{}\PYZbs{}partial z\PYZcb{}(z(t,z\PYZus{}0)) : the first part of the rhs of the variational equation }
    \PY{n}{ForwardDiff}\PY{o}{.}\PY{n}{jacobian!}\PY{p}{(}\PY{n}{dHvec}\PY{p}{,}\PY{n}{Hvec!}\PY{p}{,}\PY{n}{zpoint}\PY{p}{,}\PY{n}{z}\PY{p}{)} 
    \PY{n}{δz} \PY{o}{=} \PY{n}{flow\PYZus{}jacobi}\PY{p}{(}\PY{p}{(}\PY{n}{t0}\PY{p}{,}\PY{n}{τ}\PY{p}{)}\PY{p}{,}\PY{n}{x0}\PY{p}{,}\PY{n}{p0}\PY{p}{)}\PY{p}{[}\PY{k}{end}\PY{p}{]}
    \PY{n}{derivee\PYZus{}τ} \PY{o}{=} \PY{p}{(}\PY{n}{dHvec}\PY{o}{*}\PY{n}{δz}\PY{p}{)}\PY{p}{[}\PY{l+m+mi}{1}\PY{p}{]}    \PY{c}{\PYZsh{} \PYZsh{}derivative w.r.t. τ}

    \PY{n}{Ftau}\PY{p}{(}\PY{n}{p0}\PY{p}{)} \PY{o}{=} \PY{n}{F}\PY{p}{(}\PY{n}{t0}\PY{p}{,}\PY{n}{τ}\PY{p}{,}\PY{n}{x0}\PY{p}{,}\PY{n}{p0}\PY{p}{)}\PY{p}{[}\PY{l+m+mi}{1}\PY{p}{]} \PY{c}{\PYZsh{} First componant of the flow of the Jacobi equation at τ}
    \PY{n}{derivee\PYZus{}p0} \PY{o}{=} \PY{n}{ForwardDiff}\PY{o}{.}\PY{n}{derivative}\PY{p}{(}\PY{n}{Ftau}\PY{p}{,} \PY{n}{p0}\PY{p}{)} \PY{c}{\PYZsh{}derivative w.r.t. p0  }
    \PY{k}{return} \PY{p}{[}\PY{o}{\PYZhy{}}\PY{p}{(}\PY{l+m+mi}{1}\PY{o}{/}\PY{n}{derivee\PYZus{}τ}\PY{p}{)}\PY{o}{*}\PY{n}{derivee\PYZus{}p0}\PY{p}{]}
\PY{k}{end}

\PY{n}{p0}\PY{o}{=}\PY{o}{\PYZhy{}}\PY{l+m+mf}{0.985}
\PY{n}{τ} \PY{o}{=} \PY{l+m+mf}{0.5}
\PY{n}{x0} \PY{o}{=} \PY{l+m+mf}{1.}
\PY{n}{rhs\PYZus{}jacobi} \PY{o}{=} \PY{n}{rhs\PYZus{}path}\PY{p}{(}\PY{l+m+mf}{0.5} \PY{p}{,} \PY{l+m+mf}{1.}\PY{p}{,} \PY{n}{p0}\PY{p}{)}
\PY{n}{println}\PY{p}{(}\PY{l+s}{\PYZdq{}}\PY{l+s}{x}\PY{l+s}{0}\PY{l+s}{ }\PY{l+s}{=}\PY{l+s}{ }\PY{l+s}{\PYZdq{}}\PY{p}{,} \PY{n}{x0}\PY{p}{,} \PY{l+s}{\PYZdq{}}\PY{l+s}{,}\PY{l+s}{ }\PY{l+s}{p}\PY{l+s}{0}\PY{l+s}{ }\PY{l+s}{=}\PY{l+s}{ }\PY{l+s}{\PYZdq{}}\PY{p}{,} \PY{n}{p0}\PY{p}{)}


\PY{n+nd}{@test} \PY{n}{isapprox}\PY{p}{(}\PY{n}{rhs\PYZus{}path}\PY{p}{(}\PY{l+m+mf}{0.5} \PY{p}{,} \PY{l+m+mf}{1.}\PY{p}{,} \PY{n}{p0}\PY{p}{)}\PY{p}{,}\PY{n}{rhs\PYZus{}path\PYZus{}ana}\PY{p}{(}\PY{l+m+mf}{0.5} \PY{p}{,} \PY{l+m+mf}{1.}\PY{p}{,} \PY{n}{p0}\PY{p}{)}\PY{p}{;} \PY{n}{rtol} \PY{o}{=} \PY{l+m+mf}{1.e\PYZhy{}7}\PY{p}{)}

\PY{c}{\PYZsh{}Ftau(p0) = F(t0,τ,x0,p0)[1]\PYZsh{}flow\PYZus{}jacobi((t0,τ),x0,p0)[end]\PYZsh{}F(t0,τ,x0,p0)[end][1]}


\PY{k}{function} \PY{n}{conj\PYZus{}point}\PY{p}{(}\PY{n}{p0span}\PY{p}{,} \PY{n}{τ0}\PY{p}{)}
    \PY{n}{pb} \PY{o}{=} \PY{n}{ODEProblem}\PY{p}{(}\PY{n}{rhs\PYZus{}path}\PY{p}{,}\PY{p}{[}\PY{n}{τ0}\PY{p}{]}\PY{p}{,}\PY{n}{p0span}\PY{p}{,}\PY{p}{[}\PY{l+m+mf}{1.}\PY{p}{]}\PY{p}{)}
    \PY{n}{sol} \PY{o}{=} \PY{n}{DifferentialEquations}\PY{o}{.}\PY{n}{solve}\PY{p}{(}\PY{n}{pb}\PY{p}{,} \PY{n}{reltol} \PY{o}{=} \PY{l+m+mf}{1e\PYZhy{}8}\PY{p}{,} \PY{n}{abstol} \PY{o}{=} \PY{l+m+mf}{1e\PYZhy{}8}\PY{p}{)}
    \PY{k}{return} \PY{n}{sol}
\PY{k}{end}

\PY{c}{\PYZsh{} conjugate point}

\PY{n}{println}\PY{p}{(}\PY{l+s}{\PYZdq{}}\PY{l+s}{p}\PY{l+s}{0}\PY{l+s}{ }\PY{l+s}{=}\PY{l+s}{ }\PY{l+s}{\PYZdq{}}\PY{p}{,} \PY{n}{p0}\PY{p}{)}

\PY{n}{p0span} \PY{o}{=} \PY{p}{(}\PY{n}{p0}\PY{p}{,} \PY{o}{\PYZhy{}}\PY{l+m+mf}{0.5}\PY{p}{)}
\PY{n}{sol} \PY{o}{=} \PY{n}{conj\PYZus{}point}\PY{p}{(}\PY{n}{p0span}\PY{p}{,}\PY{n}{τ0}\PY{p}{)}
\PY{n}{plt\PYZus{}conj\PYZus{}point} \PY{o}{=} \PY{n}{plot}\PY{p}{(}\PY{n}{sol}\PY{p}{,}\PY{n}{xlabel} \PY{o}{=} \PY{n}{L}\PY{l+s}{\PYZdq{}}\PY{l+s}{p}\PY{l+s}{\PYZbs{}}\PY{l+s}{\PYZus{}}\PY{l+s}{0}\PY{l+s}{\PYZdq{}}\PY{p}{,} \PY{n}{ylabel} \PY{o}{=} \PY{n}{L}\PY{l+s}{\PYZdq{}}\PY{l+s+se}{\PYZbs{}t}\PY{l+s}{a}\PY{l+s}{u}\PY{l+s}{\PYZdq{}}\PY{p}{)}
\PY{n}{TT} \PY{o}{=} \PY{n}{sol}\PY{o}{.}\PY{n}{u}
\PY{n}{nb\PYZus{}t} \PY{o}{=} \PY{n}{length}\PY{p}{(}\PY{n}{sol}\PY{o}{.}\PY{n}{t}\PY{p}{)}
\PY{n}{T} \PY{o}{=} \PY{n}{zeros}\PY{p}{(}\PY{n}{nb\PYZus{}t}\PY{p}{)}
\PY{n}{X} \PY{o}{=} \PY{n}{zeros}\PY{p}{(}\PY{n}{nb\PYZus{}t}\PY{p}{)}
\PY{k}{for} \PY{n}{i} \PY{k+kp}{in} \PY{l+m+mi}{1}\PY{o}{:}\PY{n}{nb\PYZus{}t}
    \PY{c}{\PYZsh{}println(T[i])}
    \PY{c}{\PYZsh{}println(sol.t[i])}
    \PY{n}{T}\PY{p}{[}\PY{n}{i}\PY{p}{]} \PY{o}{=} \PY{n}{TT}\PY{p}{[}\PY{n}{i}\PY{p}{]}\PY{p}{[}\PY{l+m+mi}{1}\PY{p}{]}
    \PY{n}{X}\PY{p}{[}\PY{n}{i}\PY{p}{]} \PY{o}{=} \PY{n}{x}\PY{p}{(}\PY{n}{T}\PY{p}{[}\PY{n}{i}\PY{p}{]}\PY{p}{[}\PY{l+m+mi}{1}\PY{p}{]}\PY{p}{,}\PY{n}{x0}\PY{p}{,}\PY{n}{sol}\PY{o}{.}\PY{n}{t}\PY{p}{[}\PY{n}{i}\PY{p}{]}\PY{p}{)}
\PY{k}{end}

\PY{n}{plot!}\PY{p}{(}\PY{n}{plt\PYZus{}x}\PY{p}{,}\PY{n}{T}\PY{p}{,}\PY{n}{X}\PY{p}{,}\PY{n}{linewidth}\PY{o}{=}\PY{l+m+mi}{3}\PY{p}{)}
\PY{n}{plot}\PY{p}{(}\PY{n}{plt\PYZus{}x}\PY{p}{,}\PY{n}{plt\PYZus{}conj\PYZus{}point}\PY{p}{,}\PY{n}{layout} \PY{o}{=} \PY{p}{(}\PY{l+m+mi}{2}\PY{p}{,}\PY{l+m+mi}{1}\PY{p}{)}\PY{p}{)}
\end{Verbatim}
\end{tcolorbox}

    \begin{Verbatim}[commandchars=\\\{\}]
x0 = 1.0, p0 = -0.985
p0 = -0.985
    \end{Verbatim}

    \begin{center}
    \adjustimage{max size={0.9\linewidth}{0.9\paperheight}}{output_11_1.pdf}
    \end{center}
    { \hspace*{\fill} \\}
    
    \begin{tcolorbox}[breakable, size=fbox, boxrule=1pt, pad at break*=1mm,colback=cellbackground, colframe=cellborder]
\prompt{In}{incolor}{44}{\boxspacing}
\begin{Verbatim}[commandchars=\\\{\}]

\end{Verbatim}
\end{tcolorbox}

    \begin{tcolorbox}[breakable, size=fbox, boxrule=1pt, pad at break*=1mm,colback=cellbackground, colframe=cellborder]
\prompt{In}{incolor}{ }{\boxspacing}
\begin{Verbatim}[commandchars=\\\{\}]

\end{Verbatim}
\end{tcolorbox}


    % Add a bibliography block to the postdoc
    
    
    
\end{document}
